\documentclass{article}
    \usepackage{subcaption}
    \usepackage{amsmath, amssymb}
    \usepackage{graphicx, float}
    \usepackage[hidelinks]{hyperref}
    \usepackage[bottom]{footmisc}
    \usepackage[margin=.8in, tmargin=.8in]{geometry}
    \usepackage{esint} % for double line integrals

    \renewcommand{\baselinestretch}{1.2}
    \newcommand{\eps}{\epsilon}
    \newcommand{\der}{\partial}
    \newcommand{\del}{\nabla}
    \newcommand{\bm}[1]{\mathbf{#1}}
    \newcommand{\vf}[1]{\vec{\mathbf{#1}}}
    \newcommand{\norm}{\hat{\bm{n}}}
    
    \setlength\parindent{0pt}
    \captionsetup{justification=centering}
    
    \title{Vector Calculus}
    \date{\today}
    \author{Traiko Dinev \textless traiko.dinev@gmail.com\textgreater}
    
\begin{document}
\maketitle

\textit{NOTE: This partially follows Engineering Mathematics 2, a second year engineering course at the University of Edinburgh and partially follows MIT's 18.02, Multivariable Calculus.}

\textit{NOTE: Note this "summary" is NOT a reproduction of the course materials nor is it copied from the corresponding courses. It was entirely written and typeset from scratch.}

\textit{License: Creative Commons public license; See README.md of repository}

\textit{NOTE: Images are to be added soon!}

\section{Introduction}

Define the following:

\begin{align*}
    \del &=
        \bigg\langle
            \frac{\partial}{\partial x},
            \frac{\partial}{\partial y},
            \frac{\partial}{\partial z}
        \bigg\rangle
            && \text{gradient vector} \\
    \del f &=
        \bigg\langle
            \frac{\partial f}{\partial x},
            \frac{\partial f}{\partial y},
            \frac{\partial f}{\partial z}
        \bigg\rangle
            && \text{vector derivative}\\
    % 
    \vf{F} &=
        \langle
            f(x, y, z), g(x, y, z), h(x, y, z)
        \rangle =
        f(...) \mathbf{i} + g(...)\mathbf{j} + h(...) \mathbf{k}
            && \text{Vector field}\\
    % 
    \del \cdot \vf{F} &=
        \frac{\der f(x, y, z)}{\der x} +
        \frac{\der g(x, y, z)}{\der y} +
        \frac{\der h(x, y, z)}{\der z}
            && \text{Divergence (dot product)} \\
    % 
    \del \times \vf{F} &=
        \begin{pmatrix}
            i & j & k \\
            \frac{\der}{\der x} &
                \frac{\der}{\der x} &
                    \frac{\der}{\der x} \\
            f(...) & g(...) & h(...) 
        \end{pmatrix}
            && \text{Curl (cross product)} \\
\end{align*}

Gradients are just the derivatives in each dimension. Vector fields define a vector at each point in space, as defined by (x, y, z).
\vskip 0.1in
Here divergence gives the change in the vector field outwards (source of flux). This can also be defined as the limit of the net flow. Curl defines a rotational field (3D). It is the limit of the path integral. These two are neat relations, but I've never actually had to use them:

\begin{align*}
    \del \cdot \vf{F} &=
        \lim_{V \to p} \iint
            \frac{\vf{F} \cdot \norm}{|V|} dS \\
    (\del \times \vf{F}) \cdot \norm &=
        \lim_{A \to 0}\frac{1}{|A|} \int_c \vf{F} \cdot d\vf{r}
\end{align*}

\section{Double Integrals}
On to double (and triple) integrals. A double integral is just two single integrals stacked together. These are good to know:

\begin{align*}
    &\iint_R dA && \text{Area of R} \\
    &\iint_R \rho\ dA && \text{Mass, where $\rho$ is the density} \\
    &\bar{x} = \frac{1}{M}\iint_R x\ \rho\ dA && \text{Center of mass} \\
    &\iint_R r^2\ \rho\ dA && \text{Moment of intertia, $r$ is the distance from the axis}
\end{align*}

\subsection{Change of Variables}
Let:

\begin{align*}
    u &= u(x, y) \\
    v &= v(x, y) 
\end{align*}

then:
\begin{equation*}
dx dy = \bigg|\frac{\der(u, v)}{\der(x, y)}\bigg|^{-1} du dv = \begin{pmatrix} \frac{\der u}{\der x} & \frac{\der u}{\der y} \\ \frac{\der v}{\der x} & \frac{\der v}{\der y} \end{pmatrix} du dv
\end{equation*}

This is derived from the following, which is just from the definition of derivatives:
\begin{equation*}
    \begin{pmatrix}\Delta u \\ \Delta v\end{pmatrix} \approx 
        \begin{pmatrix}
            \frac{\der u}{\der x} & \frac{\der u}{\der y} \\
             \frac{\der v}{\der x} & \frac{\der v}{\der y}
        \end{pmatrix}
        \begin{pmatrix} \Delta x \\ \Delta y \end{pmatrix}
\end{equation*}

Or you could consider a parallelogram.. check out MIT's derivation for this one.

\section{Line Integration}
Consider a vector field $\vec{\mathbf{F}}$ and a path $c$. This could be a line or some parametric curve as defined by:

\begin{align}
    x &= x(t) \\
    y &= y(t) \\
    \vf{r} &= (x(t), y(t)) \\
    \frac{d\vec{r}}{dt} &= \langle \frac{dx}{dt}, \frac{dy}{dt} \rangle
\end{align}

for $t$ between 0 and 1. Or any other limit, but you can always scale it. You could also have more parameters.
\vskip 0.1in
Then the line (or path) integral is defined as follows:
\begin{equation*}
    \int_c \vf{F} \cdot d\vf{r} =
        \int_c Mdx + Ndy + Pdz = \int_c \vf{F} \cdot \vf{T} ds
\end{equation*}

Here $\vf{F} = (M, N, P)$ but the summation is not three separate integrals. Change of variables is needed. $\vf{T}$ is a tangent vector to the curve. $ds$ is the change along the curve.
\vskip 0.1in
Perhaps a better way to view this is by considering that work is force times distance: $W = \vf{F} \cdot \Delta \vf{r}$. Note this is a dot product between the force vector and the r defined above. When we sum, we obtain the integral:
\begin{equation}
    W = \lim_{i \to \infty} \sum_i \vf{F} \cdot \Delta \vf{r}_i = \int_C \vf{F} \cdot d\vf{r}
\end{equation}

Now we can break this up using the parametrization of $\vf{r}(t)$ above to arrive at:

\begin{equation}
    \int_c \vf{F} \cdot \Delta \vf{r} = \int_{0}^{1} (\vf{F} \cdot \frac{d\vf{r}}{dt})\ dt
\end{equation}

If you want a more rigorous derivation, you can start with the definition of derivatives and take the limit. The above is how we usually calculate line integrals. Note the derivative of r is just a vector, as defined above in (4). So path integrals are in reality just single variable integrals. Neat-o.
\vskip 0.1in
We can also do the integral in normal form. As opposed to a dot product with the tangent vector $\vf{T}$ we consider the (perpendicular) normal vector $\norm$. This is defined as the \textbf{flux} of the vector field $\vf{F}$. The flux measures how much "stuff" would be displaced by the vector field for a unit of time. It actually technically measures the force through a curve. Useful for magnetic fields:

\begin{equation}
    \int_c \vf{F} \cdot \norm\ ds \approx \sum_i (\vf{F} \cdot \norm)\ \Delta \vec{s}_i
\end{equation}

\textit{TODO: Insert parallelogram picture of normal and tangential path integrals and derivation.}

\subsection{Gradient Field Theorem}
\textit{A.k.a. the fundamental theorem of (vector) calculus}
If $\vf{F}$ is a gradient field, i.e. $\vf{F} = \nabla f$.

\begin{equation}
    \int_c \vf{F} \cdot d\vf{r} = \int \nabla f \cdot d\vf{r} = f(p_1) - f(p_0)
\end{equation}

where $p_0$ and $p_1$ are the endpoints of the curve $c$. A.k.a Stoke's theorem in 2D. \textit{TODO: add intuition}

\subsection{Green's Theorem}
This, where $c$ is counter-clockwise:

\begin{equation}
    \oint_c \vf{F} \cdot d\vf{r} = \iint_R (\nabla \times \vf{F})\ dA
\end{equation}

where $R$ is the region bounded by $c$.
\vskip 0.1in
Green's theorem also works for normal form (flux) integrals. Also known as the divergence theorem when we do in 3D.

\begin{equation}
    \oint_c \vf{F} \cdot \norm\ ds = \iint_R (\nabla \cdot \vf{F})\ dA
\end{equation}

\section{Triple Integration}
Area integrals become volume integrals:

\begin{align*}
    &\iiint_R dV && \text{volume} \\
    &dV = dx\ dy\ dz \\
    &\iiint_R \rho\ dV && \text{Mass} \\
    &\hat{f} = \frac{1}{\text{Volume}(R)}
        \iiint_R f\ dV && \text{Center of mass} \\
    &\iiint_R (\text{distance to axis})^2\delta\ dV
        && \text{Moment of inertia} \\
    &\mathbb{I}_z = \iiint_R (x^2 + y^2)^2\delta\ dV
        && \text{Moment of inertia around z-axis} \\
\end{align*}

\subsection{Cylindrical Coordinates}
\textit{TODO: add image}

\begin{align*}
    x &= \rho\ \cos \phi \\
    y &= \rho\ \sin \phi \\
    z &= z \\
    dV &= r\ dr\ d\theta\ dz
\end{align*}

\subsection{Spherical Coordinates}
\textit{TODO: add image}

\begin{align*}
    x = r\ \cos\theta = \rho \sin\phi\ \cos\theta\\
    y = r\ \sin\theta = \rho \sin\phi\ \sin\theta\\
    z = \rho \cos\theta \\
    dV = \rho^2 \sin\phi\ d\rho\ d\phi\ d\theta
\end{align*}

\section{Surface Integrals}
\textit{TODO: add image}
Natural extension to normal integration (see path integral) to three dimensions. This gives us the amount of "stuff" that goes through a patch in 3D. A patch (surface) would be the equivalent of a parametric curve in 2D.

\begin{equation}
    \iint_S \vf{F} \cdot d\vf{S} = \iint_S \vf{F} \cdot \norm\ dS
\end{equation}

Here $d\vf{S} = \norm\ dS$. Now these are a pain to calculate, but there are shortcuts:

\subsection{On a sphere}
\begin{equation*}
    dS = r^2\ \sin\phi\ d\phi\ d\theta 
\end{equation*}

Here the normal vector is $\norm = \pm \frac{1}{a} \langle x, y, z \rangle$.

\subsection{Horizontal plane}
Three separate cases, we will consider when $z = a$. Hence:
\begin{align*}
    z &= a \\
    \norm &= \pm \mathbf{\hat{k}} \\
    dS &= dx\ dy
\end{align*}

\subsection{Cylinder of $r = a$ centered on the z axis}
\begin{align*}
    \norm &= \pm \langle x, y, 0, \rangle \frac{1}{a} \\
    dS &= a\ dz\ d\theta
\end{align*}

\subsection{Graph of function}
\begin{align*}
    z &= f(x, y) \\
    \norm\ dS &= \pm \langle -f_x, -f_y, 1\rangle\ dx\ dy
\end{align*}

\subsection{Parametric Surface}
\begin{align*}
    x &= x(u, v) \\
    y &= y(u, v) \\
    z &= z(u, v) \\
    \norm\ dS &= \pm
        \bigg(\frac{\der \vf{r}}{\der u}
            \times \frac{\der \vf{r}}{\der v} \bigg) du\ dv
\end{align*}

here $\times$ is a cross product. Why? Stay tuned.. I might add a derivation. Check MIT's course.

\subsection{Slanted plane with normal $\mathbf{\hat{N}}$}
\textit{TODO}

\section{Divergence Theorem}
This is the 3D analogue of Green's theorem for flux in 2D. If:

\begin{itemize}
    \item $S$ encloses $D$ in space, where $S$ is a closed surface
    \item $\norm$ is the normal vector pointing
        \textit{outwards} and $\vf{F}$ is defined and differentiable within $D$
\end{itemize}

Then:
\begin{equation}
    \oiint_S \vf{F} \cdot d\vf{S} = \iiint_D \nabla \cdot \vf{F}\ dV
\end{equation}

This means the total stuff escaping $S$ is the net sum of all the sources within it.

\section{Stoke's Theorem}
If $C$ is a closed curve and $S$ is any surface bounded by C.

\begin{equation}
    \oint_C \vf{F} s\vf{r} = \iint_S (\nabla \times \vf{F}) \cdot \norm \ dS
\end{equation}

If I walk along $C$ with $S$ to my left, then $\norm$ is pointing up.

\end{document}
